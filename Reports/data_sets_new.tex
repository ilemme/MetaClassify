% das Papierformat zuerst
\documentclass[a4paper, 11pt]{scrartcl}
\usepackage{color}
\usepackage{hyperref}
\usepackage{numprint}
% hier beginnt das Dokument
\begin{document}
\begin{flushleft}


% Kapitelueberschrift
\section{Datensets}


% Ueberschrift eines Abschnittes
\subsection{MetaPhyler}

\begin{tabular}{ccc}
& Sequenzen & durchschnittliche Basenpaare\\
Original&73086&300\\
Rechercheergebnis&40039&644,51\\
Bazinet&73086&300\\
&&\\
\end{tabular}
\linebreak
\color{red}
\textbf{Anmerkungen zu unserer Recherche:}\linebreak
Link: \url{http://metaphyler.cbcb.umd.edu/#download}
\linebreak
\texttt{MetaPhylerSRVO.115.tar.gz(30MB)}
\linebreak
\texttt{Markers.fna wurde als Datenset verwendet.}
\color{black}
% Ueberschrift eines Abschnittes
\subsection{PhymmBL}

\begin{tabular}{ccc}
& Sequenzen & durchschnittliche Basenpaare \\
Original&80215&243\\
Rechercheergebnis&73252&204,46\\
Bazinet&80215&243\\
&&\\
\end{tabular}
\linebreak
\color{red}
\textbf{Anmerkungen zu unserer Recherche:}\linebreak
Link: \url{www.cbcb.umd.edu/software/phymm/exp_datasets/}
\linebreak
Besteht aus mehreren Textdateien aufgeteilt in 100bp, 200bp, 400bp, 800bp und 1000bp.
\color{black}

% Ueberschrift eines Abschnittes
\subsection{RAIphy}
\begin{tabular}{ccc}
& Sequenzen & durchschnittliche Basenpaare \\
Original&477000&238\\
Rechercheergebnis&477000&233,33\\
Bazinet477000&233,33\\
&&\\
\end{tabular}
\linebreak
\color{red}
\textbf{Anmerkungen zu unserer Recherche:}\linebreak
Link: \url{http://bioinfo.unl.edu/raiphy.php}
\linebreak
Genutzt Simple Test(36,6MB).
\linebreak
Besteht aus drei Fastadateien aufgeteilt in 100bp, 200bp und 400bp.
\color{black}
\newpage

% Ueberschrift eines Abschnittes
\subsection{CARMA}
\begin{tabular}{ccc}
& Sequenzen & durchschnittliche Basenpaare \\
Original&25000&265\\
Rechercheergebnis&25000&265\\
&&\\
\end{tabular}
\linebreak
\color{red}
\textbf{Anmerkungen zu unserer Recherche:}\linebreak
Link: \url{http://wwww.cebitec.uni-bielefeld.de/webcarma.cebitec.uni-bielefeld.de/download/simulated_metagenome_454_265bp.fna}
\color{black}

% Ueberschrift eines Abschnittes
\subsection{PhyloPythia}
\begin{tabular}{ccc}
& Sequenzen & durchschnittliche Basenpaare \\
Original&124941&961\\
Rechercheergebnis&114457&969,09\\
Bazinet:&\\
SIMLC& 97497&951,96\\
SIMMC& wie Recherche\\
SIMHC&116771&949,51\\
&&\\
\end{tabular}
\linebreak
\color{red}
\textbf{Anmerkungen zu unserer Recherche:}\linebreak
Link: \url{http://fames.jgi-psf.org/Retrieve_data.html}
\linebreak
SimMC Dataset sequence file runtergeladen.
\linebreak
Besteht aus 113 Fastadateien.
\color{black}

% Ueberschrift eines Abschnittes
\subsection{FACS}
\begin{tabular}{ccc}
& Sequenzen & durchschnittliche Basenpaare \\
Original&100000&269\\
Rechercheergebnis&&\\
Bazinet&100000&269\\
&&\\
\end{tabular}
\linebreak
\color{red}
\textbf{Anmerkungen zu unserer Recherche:}\linebreak
Nicht gefunden. 
\linebreak
\linebreak
\noindent\rule{\textwidth}{1pt}
\newpage
\textbf{Email an Autoren geschrieben, warten auf Antwort. --$>$ Antwort erhalten (11.11.2015) mit dem Link \url{http://molevol.umiacs.umd.edu/labfiles/sequence_classification_data_sets.
zip} \linebreak
Folgende Datensaetze nun vollstaendig:\linebreak
- CARMA\linebreak
- FACS\linebreak
- MetaPhyler\linebreak
- PHymmBL\linebreak
Bei RAIphy ist Anzahl der Sequenzen gleich, durchschnittliche Laenge unterscheidet sich.\linebreak
Bei PhyloPythia stimmen die Laengen ungefaehr ueberein. Die Anzahl der Sequenzen jedoch nicht.}
% Kapitel soll auf naechster Seite beginnen
\newpage


\end{flushleft}
% das ist wohl jetzt das Ende des Dokumentes
\end{document}